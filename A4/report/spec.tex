\documentclass[12pt]{article}

\usepackage{graphicx}
\usepackage{paralist}
\usepackage{amsfonts}
\usepackage{amsmath}
\usepackage{hhline}
\usepackage{booktabs}
\usepackage{multirow}
\usepackage{multicol}

\oddsidemargin 0mm
\evensidemargin 0mm
\textwidth 160mm
\textheight 200mm
\renewcommand\baselinestretch{1.0}

\pagestyle {plain}
\pagenumbering{arabic}

\newcounter{stepnum}

%% Comments

\usepackage{color}

\newif\ifcomments\commentstrue

\ifcomments
\newcommand{\authornote}[3]{\textcolor{#1}{[#3 ---#2]}}
\newcommand{\todo}[1]{\textcolor{red}{[TODO: #1]}}
\else
\newcommand{\authornote}[3]{}
\newcommand{\todo}[1]{}
\fi

\newcommand{\wss}[1]{\authornote{blue}{SS}{#1}}

\title{Assignment 4 Specification}
\author{SFWR ENG 2AA4}

\begin {document}

\maketitle

Freecell is a popular csolitaire card game that utilizes a set of a full set of 52 cards. Unlike other solitaire games, the chances of getting an unsolvable problem is highly unlikely. This c++ implementation of Freecell  simulates the different states a player can find themselves in while playing the game. This Document will provide the descriptive specifications behind the program, using proper modularization and adhering to other software engineering principles.

The implementation uses a minimal interface which can clearly be observed based on the use of frequent methods that are aimed to do one job. The interface is general, and allows users to move cards from pile to pile just as one would during a game.

\newpage


\section* {Card ADT Module}

\subsection*{Template Module}

Card

\subsection* {Uses}

N/A

\subsection* {Syntax}

\subsubsection* {Exported Types}

Card = ?\\
Suit = \{C, D, S, H\}\\
Colour = \{Red, Black\}\\
Rank = \{Empty, ace, two, three, four, five, six, seven , eight, nine, ten, jack, queen, king\}\\

\subsubsection* {Exported Access Programs}

\begin{tabular}{| l | l | l | l |}
\hline
\textbf{Routine name} & \textbf{In} & \textbf{Out} & \textbf{Exceptions}\\
\hline
Card& Rank, Suit & Card & \\
\hline
getRank & ~ & Rank & ~\\
\hline
getColour & ~ & Colour & ~\\
\hline
getSuit & ~ & Suit & ~\\
\hline
\end{tabular}

\subsection* {Semantics}

\subsubsection* {State Variables}

rank: Rank\\
suit: Suit\\
colour: Colour\\

\subsubsection* {State Invariant}

None

\subsubsection* {Assumptions}

The constructor Card is called for each object instance before any other
access routine is called for that object.  The constructor cannot be called on
an existing object.

\subsubsection* {Access Routine Semantics}

Card(r, s):
\begin{itemize}
\item transition: $rank, suit, colour := r, s,(s = D \lor s = H \Rightarrow  Red \| True \Rightarrow  Black)$
\item output: $out := \mathit{self}$
\item exception: None
\end{itemize}

\noindent getColour():
\begin{itemize}
\item output: $out := colour$
\item exception: None
\end{itemize}

\noindent getRank():
\begin{itemize}
\item output: $out := rank$
\item exception: None
\end{itemize}

\noindent getSuit():
\begin{itemize}
\item output: $out := suit$
\item exception: None
\end{itemize}


\newpage

\section* {Tableau ADT Module}

\subsection*{Template Module}

Tableau

\subsection* {Uses}

Card

\subsection* {Syntax}

\subsubsection* {Exported Types}

Tableau = ?

\subsubsection* {Exported Access Programs}

\begin{tabular}{| l | l | l | l |}
\hline
\textbf{Routine name} & \textbf{In} & \textbf{Out} & \textbf{Exceptions}\\
\hline
Tableau & seq of Card & Tableau & \\
\hline
addCard & Card & ~ & invalid\_move\\
\hline
removeCard & ~ & ~ & not\_available\\
\hline
topCard & ~ & Card & not\_available\\
\hline
\end{tabular}

\subsection* {Semantics}

\subsubsection* {State Variables}

$tab$: seq of Card\\

\subsubsection* {State Invariant}

None

\subsubsection* {Assumptions}

The constructor Tableau is called for each object instance before any other
access routine is called for that object.  The constructor cannot be called on
an existing object.

\subsubsection* {Access Routine Semantics}

Tableau($s$):
\begin{itemize}
\item transition: $tab := s$
\item output: $out := \mathit{self}$
\item exception: None
\end{itemize}

\noindent addCard(c):
\begin{itemize}
\item transition: $tab := tab || < c > $
\item exception: $(c.rank + 1 \neq tab_{|tab| - 1}.rank \;\lor\; c.colour = tab_{|tab| - 1}.colour \Rightarrow \mbox{invalid\_move}) $\\ 
\end{itemize}

\noindent removeCard():
\begin{itemize}
\item transition: $tab := [0 .. tab_{|tab|-2}] $
\item exception: ($|tab| = 0 \Rightarrow not\_available$)
\end{itemize}

\noindent topCard():
\begin{itemize}
\item output: $out := tab_{|tab|-1}$
\item exception: ($|tab| = 0 \Rightarrow not\_available$)
\end{itemize}

\newpage

\section* {Foundation ADT Module}

\subsection*{Template Module}

Foundation

\subsection* {Uses}

Card

\subsection* {Syntax}

\subsubsection* {Exported Types}

Foundation = ?

\subsubsection* {Exported Access Programs}

\begin{tabular}{| l | l | l | l |}
\hline
\textbf{Routine name} & \textbf{In} & \textbf{Out} & \textbf{Exceptions}\\
\hline
Foundation & Suit & Foundation & \\ 
\hline
addCard & Card  & ~ & invalid\_move\\ 
\hline
removeCard & ~ & &not\_available  \\
\hline
isFull & ~ & $\mathbb{B}$ & ~\\
\hline
topCard & ~ & Card & not\_available\\
\hline
\end{tabular}

\subsection* {Semantics}

\subsubsection* {State Variables}

pile: sequence of Card
suit: Suit

\subsubsection* {State Invariant}

None

\subsubsection* {Assumptions}

\begin{itemize}
\item The constructor Foundation is called for each object instance before any other
access routine is called for that object.  The constructor cannot be called on
an existing object.
%\item Appends will not be made while iterating through the list of lines.
\end{itemize}

\subsubsection* {Access Routine Semantics}

Foundation($s$):
\begin{itemize}
\item transition: $suit :=  s$
\item output: $out := \mathit{self}$
\item exception: None
\end{itemize}

\noindent addCard(c):
\begin{itemize}
\item transition: $pile := pile || < c > $
\item exception: $(c.rank + 1 \neq pile_{|pile| - 1}.rank \,\lor\, c.colour \neq pile_{|pile| - 1}.colour \Rightarrow \mbox{invalid\_move}) $\\ 
\end{itemize}

\noindent removeCard():
\begin{itemize}
\item transition: $pile := [0 .. pile_{|pile|-2}] $
\item exception: $|pile| = 0 \Rightarrow not\_available$
\end{itemize}

\noindent topCard():
\begin{itemize}
\item output: $out := pile_{|pile|-1}$
\item exception: $|pile| = 0 \Rightarrow not\_available$
\end{itemize}

\noindent isFull():
\begin{itemize}
\item output: $out := (|pile| = 0 \Rightarrow True | False)$
\item exception: None
\end{itemize}

\noindent getSuit():
\begin{itemize}
\item output: $out := suit$
\item exception: None
\end{itemize}

\newpage

\section* {FreeCell ADT Module}

\subsection*{Template Module}

FreeCell

\subsection* {Uses}

Card

\subsection* {Syntax}

\subsubsection* {Exported Types}

FreeCell = ?

\subsubsection* {Exported Access Programs}

\begin{tabular}{| l | l | l | l |}
\hline
\textbf{Routine name} & \textbf{In} & \textbf{Out} & \textbf{Exceptions}\\
\hline
FreeCell & ~ & FreeCell & \\ 
\hline
addCard & Card  & ~ & full\\ 
\hline
removeCard & Rank, Suit & ~ & not\_available \\
\hline
isFull & ~ & $\mathbb{B}$ & ~\\
\hline
searchCard & Rank, Suit & Card & not\_available\\
\hline
\end{tabular}

\subsection* {Semantics}

\subsubsection* {State Variables}

cells: sequence of Card

\subsubsection* {State Invariant}

None

\subsubsection* {Assumptions}

\begin{itemize}
\item The constructor FreeCell is called for each object instance before any other
access routine is called for that object.  The constructor cannot be called on
an existing object.
%\item Appends will not be made while iterating through the list of lines.
\end{itemize}

\subsubsection* {Access Routine Semantics}

FreeCell():
\begin{itemize}
\item transition: None
\item output: $out := self$
\item exception: None
\end{itemize}

\noindent addCard(c):
\begin{itemize}
\item transition: $cells := cells || < c > $
\item exception: $( isFull() \Rightarrow \mbox{full}) $\\ 
\end{itemize}

\noindent removeCard(rank, suit):
\begin{itemize}
\item transition: $cells := (\forall i :\mathbb{Z}\,\,|\,\,0 <= i < 4 : (cells_i.rank = rank \land cells_i.suit = suit) \Rightarrow  cells - \{cells_i\}) $
\item exception:  $\lnot(\exists i :\mathbb{Z}\,\,|\,\,0 <= i < 4 : (cells_i.rank = rank \land cells_i.suit = suit)) \Rightarrow  not\_available $
\end{itemize}

\noindent searchCard(rank, suit):
\begin{itemize}
\item output: $cells := (\forall i :\mathbb{Z}\,\,|\,\,0 <= i < 4 : (cells_i.rank = rank \land cells_i.suit = suit) \Rightarrow  cells_i) $
\item exception:  $\lnot(\exists i :\mathbb{Z}\,\,|\,\,0 <= i < 4 : (cells_i.rank = rank \land cells_i.suit = suit)) \Rightarrow  not\_available $
\end{itemize}
\noindent isFull():
\begin{itemize}
\item output: $out := (|cells| = 4 \Rightarrow True | False)$
\item exception: None
\end{itemize}


\newpage

\section* {Setup ADT Module}

\subsection*{Template Module}

Setup

\subsection* {Uses}

Card, FreeCell, Tableau, Foundation

\subsection* {Syntax}

\subsubsection* {Exported Types}

Setup = ?

\subsubsection* {Exported Access Programs}

\begin{tabular}{| l | l | l | l |}
\hline
\textbf{Routine name} & \textbf{In} & \textbf{Out} & \textbf{Exceptions}\\
\hline
Setup & seq (seq of Card)) & Setup & \\ 
\hline
tabToTab & $\mathbb{Z},\mathbb{Z}$  & ~ &\\ 
\hline
tabToFound & $\mathbb{Z}$  & ~ &\\ 
\hline
freeToTab & Rank, Suit, $\mathbb{Z}$  & ~ & ~ \\ 
\hline
tabToFree & $\mathbb{Z}$ & ~ & full \\
\hline
freeToFound & Rank, Suit & ~ & \\
\hline
foundToTab & Suit, $\mathbb{Z}$ & ~ & \\
\hline
winningGame & ~ & bool & \\
\hline
\end{tabular}

\subsection* {Semantics}

\subsubsection* {State Variables}

board: seq of Tableau\\
founds: seq of Foundation\\
free: FreeCell\\

\subsubsection* {State Invariant}

None

\subsubsection* {Assumptions}

\begin{itemize}
\item The constructor Setup() is called for each object instance before any other
access routine is called for that object.  The constructor cannot be called on
an existing object.
%\item Appends will not be made while iterating through the list of lines.
\end{itemize}

\subsubsection* {Access Routine Semantics}

Setup(s):
\begin{itemize}
\item transition: ($\forall i : \mathit{Z} | 0<= i < 7 : board_i = Tableau(s_i)$)
\item output: $out := self$
\item exception: None
\end{itemize}

\noindent tabToTab(from, to):
\begin{itemize}
\item transition: $board_{to},board_{from} := board_{to}||<(board_{from}.topCard())>, board_{from}.removeCard() $
\item exception: none
\end{itemize}

\noindent tabToFound(from):
\begin{itemize}
\item transition: $(\forall i :\mathbb{Z}\,\,|\,\,0 <= i < 4 : (founds_i.getSuit() = board_{from}.topcard().getSuit()) \Rightarrow (founds_i,board_{from} := founds_i || <(board_{from}.topCard())>,board_{from}.removeCard()$
\item exception:  none
\end{itemize}

\noindent freeToTab(rank, suit, to):
\begin{itemize}
\item transition: $board_{to},free :=\\ board_{to} || <(Card(rank,suit))>,[free_0 .. free_{free-1} ] - Card(rank,suit)$
\item exception: None
\end{itemize}

\noindent tabToFree(from):
\begin{itemize}
\item transition: $free,board_{from} :=\\ 
free || <(board_{from}.topCard())>,board_{from}.removeCard()$
\item exception: $(free.isFull()) \Rightarrow full$
\end{itemize}

\newpage

\noindent freeToFound(rank,suit):
\begin{itemize}
\item transition: \\$(\forall i :\mathbb{Z}\,\,|\,\,0 <= i < 4 : (founds_i.getSuit() = free.searchCard(rank,suit)).topcard().getSuit()) \Rightarrow \\(founds_i,free := founds_i || <(free.searchCard(rank,suit)))>,free.removeCard(rank,suit)$
\item exception: none
\end{itemize}

\noindent foundToTab(rank,to):
\begin{itemize}
\item transition: \\$(\forall i :\mathbb{Z}\,\,|\,\,0 <= i < 4 : (founds_i.getSuit() = suit) \Rightarrow \\(board_{to},founds_i := board_{to} || <(founds_i.topCard())>,founds_i.removeCard()$
\item exception: none
\end{itemize}

\noindent winningGame():
\begin{itemize}
\item output: $\land$ $(i : \mathbb{N} | i \in [0 ..3] : founds_i.isFull())$
\item exception: none
\end{itemize}

\subsubsection* {Local Functions}

\begin{itemize}
\item tabTopCards: A method that outputs the top cards on the tableau to clearly
see what is available suring a specific state.
\end{itemize}

\end {document}
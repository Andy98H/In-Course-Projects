\documentclass[12pt]{article}

\usepackage{graphicx}
\usepackage{paralist}
\usepackage{listings}

\oddsidemargin 0mm
\evensidemargin 0mm
\textwidth 160mm
\textheight 200mm

\pagestyle{plain}
\pagenumbering{arabic}

\newcounter{stepnum}

\title{Assignment 1 Report}
\author{Andy Hameed and hameea1}
\date{\today}

\begin{document}

\maketitle

In assignment 1, part 1 we were asked to create a program that produced a curve and sequence ADT in python. The instructions were intentionally made ambiguous to yield different interpretations by students. This report will summarize the issues with ambiguity in a program specification and some questions on the methods for each ADT class.

\section{Testing of the Original Program}

Most of the test cases in my testSeq module used assertions to match the output of a method call with the expected output of the method. For example, the add method either appends an element to the end of the list or inserts an element in the sequence between to other elements, so I checked for these two cases and asserted the resulting sequence after an element was added or appended. For tests that were supposed to fail, I used a exceptional statements in my original code that would print out an error message. In hindsight, this is a futile way of catching errors and asserting test cases because these messages are very specific and likely not taken into consideration or foreseen by someone else.

I followed a form similar to the one provided in the assignment solution test module and counted the number of passing tests and overall number of tests, with a print statement displaying a pass or fail for each method and the ratio of passing tests to total tests. When testing my code, all tests passed. However, this result, as expected, does not make the two modules robust because there are cases that are not covered and unspecified by the assignment instructions. The testing was based on a source input file that I created with my own data, which elicits distorted test results.


\section{Results of Testing Partner's Code}

Results of Testing Partner’s Code When testing my partner’s code, only 5/10 tests passed through the test module. The passing tests were:
\begin{itemize}
	\item size
	\item indexInSeq
	\item linVal
	\item quadVal
	\item npolyVal
\end{itemize}
\section{Discussion of Test Results}

\subsection{Problems with Original Code}
My original code did not have many issues but my testSeq module was not reliable. Each test for the \texttt{SeqT} class relied on the previous test based on the order of elements in the sequence after a given test call. This gave off inaccurate testing results

\subsection{Problems with Partner's Code}
When naming the instance variable in the constructor, my partner used self.seq mean- while I used self.state. This made the test module inaccurate in testing my partner’s code because it was specifically tailored for the instance variable that I created. As a result, methods in CurveT were affected by this as well since the SeqT is imported into CurveT and used.

\subsection{Problems with Assignment Specification}

The assignment specification were made ambiguous in many ways. An example of this is seen in the instructions for quadratic interpolation, where $x_{0}$ is not specified in relation to x itself. I assumed it to be a point to the left of $x_1$ but there could have been many interpretations of this. I also found the specification for the add method to be ambiguous as well. At first I interpreted "adding" an element at index i means that the element comes after index i, but later on I decided to assume the value is being placed to be at index i and the rest of the sequence is pushed to the right instead.

\section{Answers}

\begin{enumerate}

\item For each of the methods in each of the classes, please classify it as a
  constructor, accessor or mutator.
\begin{center}
\begin{tabular}{ | c | c | c |}
\hline
Constuctors & Accessors & Mutators \\
\hline
CurveT & indexInSeq & set \\
\hline
SeqT & size & rm \\
\hline
& get & add \\
\hline
& linVal &  \\
\hline
& quadVal &  \\
\hline
& npolyVal &  \\
\hline
\end{tabular}
\end{center}

\item What are the advantages and disadvantages of using an external library
  like \texttt{numpy}?

numpy has built in functions like polyval and polyfit that evaluate and implement complex mathematical procedures. This abstraction allows developers to focus on the main implementation of their methods. Another advantage is that external libraries improves reliability since these methods have been taken consideration by the developers of the external library. Overall, external libraries improve productivity. On the other hand, external libraries like numpy could restrict implementation and customization. When functions have already been built and are abstracted from the developer, optimizing algorithms, customizing outputs and other challenges arise because the programmer may not have access to/permission to change the implementation of library functions.

\item The \texttt{SeqT} class overlaps with the functionality provided by
  Python's in-built list type.  What are the differences between \texttt{SeqT}
  and Python's list type?  What benefits does Python's list type provide over
  the \texttt{SeqT} class?

The sequence class is designed to represent and resemble a sequence. In general, using the list type is more intuitive when it comes to using methods as opposed to the SeqT class which has methods that are difficult to interpret (Ex. indexToSeq). Although SeqT uses a list as its instance variable, it has functions that are not built into the list python module. Take for example indexInSeq which returns the”position” where a value would be placed into a given sequence. This function is unique to the class and The SeqT uses customized exception statements to dodge error messages which is another advantage over Python’s list type. However, the list type has a multitude of built-in functions and methods which provides users with more control over the manipulation of lists. Python’s list type and the SeqT class both have their place in the implementation with their own advantages.

\item What complications would be added to your code if the assumption that
  $x_i < x_{i+1}$ no longer applied?

An unsorted set of data presents several challenges. In order to implement methods in the curve class especially, a sorting algorithm would be needed before indexInSeq is used or a searching algorithm to find the two numbers in the sequence for linear and quadratic interpolation.

\item Will \texttt{linVal(x)} equal \texttt{npolyVal(n, x)} for the same \texttt{x}
  value?  Why or why not?

No, because the npolyVal method takes the argument n representing the degree of the polynomial that best fits the whole data at that point. So for a call of npolVal(1, x) we have a line of best fit being used to represent the whole data. We obtain our y value from a method of regression. Meanwhile the linVal method draws a line between precisely two points to represent the data, the point before and after x.

\end{enumerate}

\newpage

\lstset{language=Python, basicstyle=\tiny, breaklines=true, showspaces=false,
  showstringspaces=false, breakatwhitespace=true}
%\lstset{language=C,linewidth=.94\textwidth,xleftmargin=1.1cm}

\def\thesection{\Alph{section}} 

\section{Code for SeqADT.py}

\noindent \lstinputlisting{../src/SeqADT.py}

\newpage

\section{Code for CurveADT.py}

\noindent \lstinputlisting{../src/CurveADT.py}

\newpage

\section{Code for testSeqs.py}

\noindent \lstinputlisting{../src/testSeqs.py}

\newpage

\section{Code for Partner's SeqADT.py}

\noindent \lstinputlisting{../partner/SeqADT.py}

\newpage

\section{Code for Partner's CurveADT.py}

\noindent \lstinputlisting{../partner/CurveADT.py}

\newpage

\section{Makefile}

\lstset{language=make}
\noindent \lstinputlisting{../Makefile}

\end{document}
